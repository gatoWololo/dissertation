
\chpt{Capybara: Low-Latency Live TCP Connection Migration}
n this work, we present Capybara. A system for live TCP connection migration.
\section{Motivation}
- Datacenter Latency Sensitive Applications
- Load balancing
\section{Background}
\subsection{Kernel Bypass IO}
Recent advancements in hardware as well as demand from datacenters have brought a new generation of technologies into the OS and networking space. Among this changes are a class of technilogies known as kernel-bypass IO.

Network IO is classically done via system calls into the OS kernel. System calls incur slight overhead as a mode switch is required to go into kernel space. As datacenter applications have become IO bound the have also become IO sensitive. This has become a large percentage of the cost for some applications. For example, in-memory key value stores like Redis \cite{redis} and Memcached \cite{memcached}.
\subsection{Programmable Switches}
For latency-critical application, like those found in the datacenter, 
\subsection{Demikernel}
\subsection{TCP Connections}
\subsection{Load Balancing}

\section{Live TCP Connection Migration}
This work is a direct successor to Prism \cite{prism}. Prism shows the feasibility of migrating TCP connection between backend servers.

Capybara reimplements the TCP handoff protocol introduced in Prism, but in a novel context: the kernel-bypass network IO.

This has the following advantages of Prism:
- We control the userspace network stack. Prism requires network support 

- P4
- Fast
- No need to change protocol
- Dynamic Load Balancing

\section{Capybara Design}
In this section we present our design for live TCP connection migration. We strive to make this migration as general as possible as to support a wide range of usescases.

We strive to make as few assumptions about the datacenter network topology as possible. We consider a connection made between a client and a server application. This connection will be migrated from an origin machine to a destination machine. The network traffic between the client, origin machine, and destination machine is mediated by a programmable P4 switch. Both origin and destination machines have DPDK-supported NICs and run instances of the Catnip network stack.

\subsection{Migration Outline}
Migration Outline

Definitions
P is a server application modified to use Catnip for low-latency kernel-bypass networking. One instance of P runs per machine M.
M is a machine involved in live TCP Migration. Either the origin or the destination.
C is the current TCP Connection we want to migrate.
B is a TCP (or UDP?) connection between the origin and destination. This connection is used exclusively for migrating network connections in/out. Applications are expected to poll this connection frequently to accept incoming migrating connections. Every instance has this connection between each other.

Outline
The origin decides to migrate an existing connection to a destination instance.
The origin concurrently:
Reroute Traffic
Messages the switch asking it to reroute packets of connection C to the destination. 
1) Serialize Connection
Calls into Catnip via the `migrate\textunderscore{}connection\textunderscore{}out` method. This has the following effects:
This connection gets tagged as `ConnectionMigratedOut`.
Additional packets for this connection are buffered by Catnip.
`migrate\textunderscore{}connection\textunderscore{}out` returns `ConnectionState`, the serialized state of the specified connection.
2) Migrates connection state
The origin uses B to send over the state of the connection to the destination.
The destination receives the incoming `ConnectionState` by polling B. It uses the `ConnectionState` to establish a new connection by calling into Catnip via `migrate\textunderscore{}connection\textunderscore{}in`. C is put in a special state `ConnectionMigratedIn`. This has the following effect:
Incoming packets will have an IP address different from the machine’s IP address. This is okay, Catnip rewrites/ignores the IP address for this connection.
Outgoing packets have their source IP address rewritten to that of the origin.
The destination sends the origin instance an END message via B. This has two effects:
The origin sends the destination any remaining buffered packets for C.
The origin removes any remaining state from its stack related to C.
The destination receives any buffered packets from O.
The connection is now fully migrated.


\subsection{Migration-aware applications}
\subsection{Migration mechanism orthogonal from migration trigger}
\subsection{Handling Program State}

\section{Implementation}
Capybara is implemented on top of the Demikernel.

DPDK
RDMA
P4 Communication
\section{Ongoing Work}
- This is ongoing work.

\subsection{Evaluation}
\section{Related Work}
- Caladan